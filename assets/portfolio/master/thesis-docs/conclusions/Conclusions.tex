\chapter{Discussion and Conclusions}

	With what has been developed throughout this thesis work, it has given us a satisfactory understanding of the knowledge necessary to apply spectral methods in solving initial value problems of nonlinear partial differential equations, which with practice it will be possible to acquire the ability to study and attack other more complex problems. \\
	
	The practice that we have obtained when developing Chapter \ref{Chapter_3} with the help of the tools examined in Chapter \ref{Chapter_2}, can be useful to construct and study some other spectral method using another family of polynomials that allows us to approximate more precisely a function, and to construct practical methods for its implementation solving this same problem or another one of interest. Although, as expected based on theory, we were able to obtain good approximations using Fourier methods to solve linear problems. But, the situation was different when we considered the non-linear problem since it was not possible to obtain the solutions directly, forcing us to develop more carefully the numerical methods that finally gave us a good numerical approximation and with practical mathematical expressions for its implementation.  \\
	
	However, we noted that very small steps in time were required to ensure that the methods worked correctly, which greatly increased the computation time, but this was due to the presence of the aliasing error caused by the nonlinear term, which manifested itself unfavorably with small values of $\alpha$, and they should be handled more carefully, possibly considering another class of polynomials. Even so, we were able to satisfactorily detect the problems that can arise when using these tools, and it is interesting to delve into this with the help of the appropriate tools that allow us to characterize and know the convergence of these methods. \\
	
	Regarding the spectral method that we developed in the Chapter \ref{Chapter_4}, the problem that arises is due to the number of calculations that are necessary to obtain the numerical approximations, but nevertheless, the implementation is easy to understand and can be studied with more detail to propose a more efficient and less expensive algorithm to calculate, mainly attacking the calculation of the integrals involved in the process. For this type of problem, it is worth investigating an implementation that allows you to write parallel code in some programming language to optimize memory and computation times. \\
	
	Even so, it was possible to successfully carry out some numerical experiments that allowed us to observe a theoretical result that characterized the stability of this method, and perhaps could allow us to investigate other characteristics of interest, such as its convergence. Thanks to the practice developed in this chapter we have acquired the ability to implement a spectral method to carry out numerical studies that may be useful for its analysis, which could also be interesting to implement these same techniques in other problems that are considered important. \\
	
	Therefore, we can conclude that spectral methods are a good option to study partial differential equations, either deterministic or stochastic since in addition to giving us the possibility of obtaining good precision in their approximations provided they are implemented correctly, they are also an excellent alternative to investigate the nature of these problems, be they physical or mathematical, opening different paths, such as some of those already mentioned, that may be interesting for developing future research. \\
	
	Finally, from a personal perspective, it considers of great importance the knowledge about the use of programming languages to build quality algorithms, that is, that they manage to have a code structure that can be a base to solve other problems. In addition, good training in scientific computing is of great advantage when producing research on the development of algorithms that allow to satisfactorily show the precision of some numerical method, since it must be considered that this depends strongly on the quality of the computation and could interfere considerably with the evaluation of our experiments. 
	
	
	